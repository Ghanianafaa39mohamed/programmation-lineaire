\documentclass[a4paper,11pt]{article}

\usepackage[T1]{fontenc}
\usepackage[utf8]{inputenc}
\usepackage{graphicx}
\usepackage{xcolor}

\usepackage{tgtermes}

\usepackage[
pdftitle={Programmation linéaire},
pdfauthor={Nicolas Mauger <nicolas.mauger@epsi.fr>},
colorlinks=true,linkcolor=blue,urlcolor=blue,citecolor=blue,bookmarks=true,
bookmarksopenlevel=2]{hyperref}
\usepackage{amsmath,amssymb,amsthm,textcomp,mathrsfs}
\usepackage{enumerate}
\usepackage{multicol}
\usepackage{pgf,tikz}
\usetikzlibrary{arrows}
\usetikzlibrary[patterns]

\usepackage{geometry}
\geometry{total={210mm,297mm},
left=25mm,right=25mm,%
bindingoffset=0mm, top=20mm,bottom=20mm}

\linespread{1.3}

\newcommand{\linia}{\rule{\linewidth}{0.5pt}}

% custom theorems if needed
\newtheoremstyle{mytheor}
    {1ex}{1ex}{\normalfont}{0pt}{\scshape}{.}{1ex}
    {{\thmname{#1 }}{\thmnumber{#2}}{\thmnote{ (#3)}}}

\theoremstyle{mytheor}
\newtheorem{theor}{Théorème}

% my own titles
\makeatletter
\renewcommand{\maketitle}{
\begin{center}
\vspace{2ex}
{\huge \textsc{\@title}}
\vspace{1ex}
\\
\linia\\
\@author \hfill \@date
\vspace{4ex}
\end{center}
}
\makeatother
%%%

% custom footers and headers
\usepackage{fancyhdr,lastpage}
\pagestyle{fancy}
\lhead{}
\chead{}
\rhead{}
\lfoot{\textit{Recherche opérationnelle}}
\cfoot{}
\rfoot{Page \thepage\ /\ \pageref*{LastPage}}
\renewcommand{\headrulewidth}{0pt}
\renewcommand{\footrulewidth}{0pt}


\begin{document}

\title{EPSI B3 - Programmation linéaire\\Chapitre 2 : Application à la 3D}

\author{Nicolas Mauger}

\date{7 décembre 2016}

\maketitle

\section{Optimisation 3D}
On cherche maintenant à maximiser ou minimliser une fonction $Z$, qu'on écrira
$\max(Z(x,y,z))$ tel que $Z(x,y,z) = ax + by + cz$. Pour faciliter l'application
 à l'algorithmique, on écrira les variables à l'aide d'indice. \\

On obtient donc le système d'équation à $m$ contraintes suivant :

\begin{equation*}
  s.c
     \begin{cases}
        a_{11}X_{1} + a_{12}X_{2} + a_{13}X_{3} \leq b_{1} \\
        a_{21}X_{1} + a_{22}X_{2} + a_{23}X_{3} \leq b_{2} \\
        \vdots                           \\
        a_{m1}X_{1} + a_{m2}X_{2} + a_{m3}X_{3} \leq b_{m} \\
     \end{cases}
\end{equation*}

\subsection{Exemple de résolution}
Soit la fonction à maximiser
$$ Z(X_{1},X_{2},X_{3}) = 10X_{1} + 13X_{2} + 12X_{3} $$

On définit les contraintes suivantes :

\begin{equation*}
  s.c
     \begin{cases}
        X_{1} + 2X_{2} + 3X_{3} \leq 120 \\
        3X_{1} + X_{2} + 2X_{3} \leq 120 \\
        X_{1} + 4X_{2} + X_{3} \leq 120 \\
        X_{1}, X_{2}, X_{3} \geq 0 \\
     \end{cases}
\end{equation*}

Ce qui nous donne trois équations cartésienne de plan:
$$ P_{1} \rightarrow X_{1} + 2X_{2} + 3X_{3} = 120 $$
$$ P_{2} \rightarrow 3X_{1} + X_{2} + 2X_{3} = 120 $$
$$ P_{3} \rightarrow X_{1} + 4X_{2} + X_{3} = 120 $$

\newpage

On trace alors les différents plans $P_{n}$ tels que :
$$ A_{n} \in P_{n}  \cap O\overrightarrow{X_{1}} $$
$$ B_{n} \in P_{n}  \cap O\overrightarrow{X_{2}} $$
$$ C_{n} \in P_{n}  \cap O\overrightarrow{X_{3}} $$

% Geogebra currently do not support 3D export

La zone de programme admissible sera l'ensemble des points dans le pavé $[ABCDEFG]$.
Comme pour la forme duale, on peut calculer la valeur de chacun de ces sommets avec la fonction économique:
\begin{center}
   \begin{tabular}{ c | c  }
     Sommet & Valeur \\ \hline
     $O(0,0,0)$ & $Z_{O} = 0$ \\
     $A(40,0,0)$ & $Z_{A} = 400$ \\
     $C(0,30,0)$ & $Z_{C} = 390$ \\
     $E(0,0,40)$ & $Z_{E} = 480$ \\
     $B(\frac{360}{11},\frac{260}{11},0)$ & $Z_{B} = 610,9$ \\
     $F(\frac{120}{7},0,\frac{240}{7})$ & $Z_{F} = 382,86$ \\
     $D(0,24,24)$ & $Z_{D} = 600$ \\
     $G(20,20,20)$ & $Z_{G} = 700$ \\
   \end{tabular}
 \end{center}

On constate alors que c'est au point $G$ qu'on obtient l'optimum de $Z$, le point $G$ est donc solution de la fonction économique $Z$.


\end{document}
