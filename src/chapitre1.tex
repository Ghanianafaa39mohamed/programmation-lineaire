\documentclass[a4paper,11pt]{article}

\usepackage[T1]{fontenc}
\usepackage[utf8]{inputenc}
\usepackage{graphicx}
\usepackage{xcolor}

\usepackage{tgtermes}

\usepackage[
pdftitle={Programmation linéaire},
pdfauthor={Nicolas Mauger <nicolas.mauger@epsi.fr>},
colorlinks=true,linkcolor=blue,urlcolor=blue,citecolor=blue,bookmarks=true,
bookmarksopenlevel=2]{hyperref}
\usepackage{amsmath,amssymb,amsthm,textcomp,mathrsfs}
\usepackage{enumerate}
\usepackage{multicol}
\usepackage{pgf,tikz}
\usetikzlibrary{arrows}
\usetikzlibrary[patterns]

\usepackage{geometry}
\geometry{total={210mm,297mm},
left=25mm,right=25mm,%
bindingoffset=0mm, top=20mm,bottom=20mm}

\linespread{1.3}

\newcommand{\linia}{\rule{\linewidth}{0.5pt}}

% custom theorems if needed
\newtheoremstyle{mytheor}
    {1ex}{1ex}{\normalfont}{0pt}{\scshape}{.}{1ex}
    {{\thmname{#1 }}{\thmnumber{#2}}{\thmnote{ (#3)}}}

\theoremstyle{mytheor}
\newtheorem{theor}{Théorème}

% my own titles
\makeatletter
\renewcommand{\maketitle}{
\begin{center}
\vspace{2ex}
{\huge \textsc{\@title}}
\vspace{1ex}
\\
\linia\\
\@author \hfill \@date
\vspace{4ex}
\end{center}
}
\makeatother
%%%

% custom footers and headers
\usepackage{fancyhdr,lastpage}
\pagestyle{fancy}
\lhead{}
\chead{}
\rhead{}
\lfoot{\textit{Recherche opérationnelle}}
\cfoot{}
\rfoot{Page \thepage\ /\ \pageref*{LastPage}}
\renewcommand{\headrulewidth}{0pt}
\renewcommand{\footrulewidth}{0pt}


\begin{document}

\title{EPSI B3 - Programmation linéaire\\Chapitre 1 : Recherche de forme duale}

\author{Nicolas Mauger}

\date{22 octobre 2016}

\maketitle

\section{Maximisation de la fonction $Z$}
On cherche à maximiser ou minimliser une fonction $Z$, qu'on écrira
$\max(Z(x,y))$ tel que $$Z(x,y) = ax + by$$

On obtient donc le système d'équation à $m$ contraintes suivant :

\begin{equation*}
  s.c
     \begin{cases}
        a_{11}x_{1} + a_{12}y \leq b_{1} \\
        a_{21}x_{1} + a_{22}y \leq b_{2} \\
        \vdots                           \\
        a_{m1}x_{1} + a_{m2}y \leq b_{m}
     \end{cases}
\end{equation*}

\subsection{Résolution}
La résolution se problème se fait en 3 étapes.
\begin{enumerate}
  \item Localiser $Z$ dans la zone de programme admissible;
  \item Tracer la droite $ \Delta : Z = 0 $ avec $ax+by=0$
  \item On fait glisser la droite $\Delta$ dans la zone de programme admissible
  et on cherche le point le plus lointain de l'origine dans la zone de programme
  admissible qui rencontre une parallèle à $\Delta$. On note ce point
  $H(x_{0},y_{0})$.
\end{enumerate}

\paragraph{Remarque}
On peut tomber sur des contraintes incompatibles, tels que $Opt(Z)$ n'existe pas. \\
On peut aussi trouver une zone admissible infinie; dans ce cas on dit que la
zone de programme admissible est une zone non bornée, et que $\max(Z)=+\infty$. \\

\begin{theor}
L'image par une fonction continu d'une zone fermée bornée est une zone fermée
bornée.
\end{theor}

\begin{theor}
L'optimum d'une fonction continu définit sur un compact (c.à.d une zone fermée
bornée) se trouve sur la frontière de cette zone.
\end{theor}

\begin{theor}
Si un optimum d'une fonction économique $Z$ est le même en deux sommets $A$ et
$B$, alors l'optimum est constant sur tout le segment $[A,B]$
\end{theor}

\subsection{Intérprétation graphique}
Calulons $\min(Z(x,y))$ avec $Z(x,y) = \frac{3}{2}x + y$ sous les contraintes suivantes:
\begin{equation*}
  s.c
    \begin{cases}
      3x + 2y \geq 6 \\
      2x - 3y \leq 0 \\
      x + y \leq 8 \\
      0 \leq y \leq 4 \\
      x \geq 1
    \end{cases}
\end{equation*}

\definecolor{ccwwff}{rgb}{0.8,0.4,1.}
\definecolor{uuuuuu}{rgb}{0.26666666666666666,0.26666666666666666,0.26666666666666666}
\definecolor{qqttcc}{rgb}{0.,0.2,0.8}
\definecolor{ttzzqq}{rgb}{0.2,0.6,0.}
\definecolor{ffqqqq}{rgb}{1.,0.,0.}
\begin{tikzpicture}[line cap=round,line join=round,>=triangle 45,x=1.0cm,y=1.0cm]
\draw[->,color=black] (-5.586350201989329,0.) -- (8.863832371063486,0.);
\foreach \x in {-4.,-2.,2.,4.,6.,8.}
\draw[shift={(\x,0)},color=black] (0pt,2pt) -- (0pt,-2pt) node[below] {\footnotesize $\x$};
\draw[->,color=black] (0.,-3.8921283811623044) -- (0.,6.950813314623452);
\foreach \y in {-2.,2.,4.,6.}
\draw[shift={(0,\y)},color=black] (2pt,0pt) -- (-2pt,0pt) node[left] {\footnotesize $\y$};
\draw[color=black] (0pt,-10pt) node[right] {\footnotesize $0$};
\clip(-5.586350201989329,-3.8921283811623044) rectangle (8.863832371063486,6.950813314623452);
\fill[color=ccwwff,fill=ccwwff,pattern=dots,pattern color=ccwwff] (1.,4.) -- (1.,1.5) -- (1.3846153846153846,0.9230769230769231) -- (4.8,3.2) -- (4.,4.) -- cycle;
\draw [line width=1.2pt,color=ffqqqq,domain=-5.586350201989329:8.863832371063486] plot(\x,{(--6.-3.*\x)/2.});
\draw [line width=1.2pt,color=ttzzqq,domain=-5.586350201989329:8.863832371063486] plot(\x,{(-0.-2.*\x)/-3.});
\draw [line width=1.2pt,color=qqttcc,domain=-5.586350201989329:8.863832371063486] plot(\x,{(--8.-1.*\x)/1.});
\draw [dash pattern=on 5pt off 5pt,domain=-5.586350201989329:8.863832371063486] plot(\x,{(--4.-0.*\x)/1.});
\draw [dash pattern=on 5pt off 5pt] (1.,-3.8921283811623044) -- (1.,6.950813314623452);
\draw [color=ccwwff] (1.,4.)-- (1.,1.5);
\draw [color=ccwwff] (1.,1.5)-- (1.3846153846153846,0.9230769230769231);
\draw [color=ccwwff] (1.3846153846153846,0.9230769230769231)-- (4.8,3.2);
\draw [color=ccwwff] (4.8,3.2)-- (4.,4.);
\draw [color=ccwwff] (4.,4.)-- (1.,4.);
\draw (5.362686813715888,3.5345440132115016) node[anchor=north west] {$VE \, = \,10.4$};
\draw (2.2222653440950113,0.9245991431887067) node[anchor=north west] {$VD \, = \,3$};
\draw (-1.3000992772365116,1.8157998305135634) node[anchor=north west] {$VC \, = \,3$};
\draw (1.2037502728666192,5.062316620054113) node[anchor=north west] {$VB \, = \,5.5$};
\draw (4.110762038664322,5.062316620054113) node[anchor=north west] {$VA \, = \,10$};
\begin{scriptsize}
\draw[color=ffqqqq] (-1.6396043009793089,6.802279866735976) node {$3x + 2y = 6$};
\draw[color=ttzzqq] (-4.780025770600185,-3.234337397660625) node {$2x - 3y = 0$};
\draw[color=qqttcc] (2.010074704255763,6.802279866735976) node {$x + y = 8$};
\draw[color=black] (-5.161968922310832,3.8316109089864536) node {$y = 4$};
\draw[color=black] (1.479598104657642,6.802279866735976) node {$x = 1$};
\draw [fill=uuuuuu] (4.,4.) circle (1.5pt);
\draw[color=uuuuuu] (4.153200166632171,4.298430316632807) node {$A$};
\draw [fill=uuuuuu] (1.,4.) circle (1.5pt);
\draw[color=uuuuuu] (1.1400930809148446,4.298430316632807) node {$B$};
\draw [fill=uuuuuu] (1.,1.5) circle (1.5pt);
\draw[color=uuuuuu] (1.1400930809148446,1.7945807665296383) node {$C$};
\draw [fill=uuuuuu] (1.3846153846153846,0.9230769230769231) circle (1.5pt);
\draw[color=uuuuuu] (1.5432552966094166,1.2216660389636589) node {$D$};
\draw [fill=uuuuuu] (4.8,3.2) circle (1.5pt);
\draw[color=uuuuuu] (4.938305534037391,3.492105885243651) node {$E$};
\end{scriptsize}
\end{tikzpicture}

On peut alors construire un tableau ou l'on calcule la valeur de chacun des sommets de la zone de programme admissible $[ABCDE]$ avec la fonction économique $Z$:

\begin{center}
   \begin{tabular}{ c | c  }
     Sommets & Valeurs \\ \hline
     $A(4;4)$ & $Z_{A} = 10$ \\
     $B(1;4)$ & $Z_{B} = 5,5$ \\
     $C(1;1,5)$ & $Z_{C} = 3$ \\
     $D(\frac{69}{50};\frac{23}{25})$ & $Z_{D} = 3$ \\
     $E(\frac{24}{5};\frac{16}{5})$ & $Z_{D} = 10,4$ \\
   \end{tabular}
 \end{center}

 On constate alors qu'il y a deux valeurs minimales: le sommet $C$ et le sommet $D$.
 Or selon le théorème numéro 3, si un optimum d'une fonction économique $Z$ est le même
 en deux sommets $A$ et $B$, alors l'optimum est constant sur tout le segment
 $[A,B]$. Donc la solution de la fonction économique $Z$ est l'ensemble
 des points sur le segment $[C,D]$.


\end{document}
